\documentclass[twocoloumn]{article}
\usepackage[utf8]{inputenc}
\usepackage{amssymb}
\usepackage{gensymb}
\usepackage{amsmath}
\usepackage{graphicx}
\usepackage{multicol}
\usepackage{tabularx}
\graphicspath{{figures/}}
\title{AI1110 - Assignment 1}
\author{Kushagra Gupta \\ \normalsize CS21BTECH1103 \\ \vspace*{20pt}\\ \vspace*{20pt} \Large ICSE 2018 Grade 10 \\ \large Q10-c}
\date{}
\begin{document}
\maketitle 
\begin{multicols}{2}
\section*{Problem Statement}


The angle of elevation from a point P of the top of a tower QR, 50m high is $60\degree$ and that of the
tower PT from a point Q is $30\degree$. Find the height of tower PT, correct to the nearest metre.

\begin{center}
\includegraphics[width=167.5pt]{figure.png}
\end{center}
\section*{Solution}

\begin{tabular}{|c|c|c|}
\hline
\textbf{Parameter} & \textbf{Symbol} & \textbf{Value} \\
\hline
QR & $h$ & $50$ \\
\hline
Angle QPR & $\angle QPR$ & $60\degree$ \\
\hline
Angle PQT & $\angle PQT$ & $30\degree$ \\
\hline
Base PQ & $d$ & ??? \\
\hline
PT & $h2$ & ??? \\
\hline
\end{tabular}

In $\Delta$PQR, using basic trigonometric equation in a right-angled triangle, we know that,

\begin{equation}
    \tan(\theta)=\frac{perpendicular}{base}
\end{equation}

\noindent Hence, 

\begin{align*}
\hspace{30pt}&\tan(\angle QPR) = \frac{h}{d} \\
&\Rightarrow d = \frac{h}{\tan(\angle QPR)} \\
&\Rightarrow d = \frac{50}{\tan(60\degree)}\, m \\
&[\because \angle QPR = 60\degree \, \& \, h = 50m] \\
&\Rightarrow d = \frac{50}{\sqrt{3}}\, m \hspace{10pt}-(1)
\end{align*}

Now in $\Delta PQT$, $\angle PQT = 30\degree$.

\begin{align*}
\therefore \tan&(\angle PQT) = \frac{h_2}{d} \\
&\Rightarrow h_2 = d \times \tan(\angle PQT)\\
&\Rightarrow h_2 = d \times \tan(30 \degree)\\
&\Rightarrow h_2 = \frac{50}{\sqrt{3}} \times \tan(30 \degree) \, m \\
&[using (1)]\\
&\Rightarrow h_2 = \frac{50}{3}\, m
\end{align*}

\noindent $\therefore h_2(PT)   \approx 17 $ metres after rounding off.

\vspace{2pt}
\noindent This can be verified by plotting QR , $\angle RPQ$ and $\angle PQT$ and approximating 
\noindent the length of PT.

\end{multicols}

\section*{Output}
\noindent The Output of the program used to verify the answer is given below:

\begin{figure}[h]
\includegraphics[width=\textwidth]{output.png}
\caption{Plot of the figure and calculated length}

\end{figure}
\end{document}
