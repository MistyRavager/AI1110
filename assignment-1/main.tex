\let\negmedspace\undefined
\let\negthickspace\undefined
\documentclass[journal,12pt,twocolumn]{IEEEtran}
\usepackage{gensymb}
\usepackage{amssymb}
\usepackage[cmex10]{amsmath}
\usepackage{amsthm}
\usepackage[export]{adjustbox}
\usepackage{bm}
\usepackage{longtable}
\usepackage{enumitem}
\usepackage{mathtools}
 \usepackage{tikz}
\usepackage[breaklinks=true]{hyperref}
\usepackage{listings}
\usepackage{color}                                            %%
\usepackage{array}                                            %%
\usepackage{longtable}                                        %%
\usepackage{calc}                                             %%
\usepackage{multirow}                                         %%
\usepackage{hhline}                                           %%
\usepackage{ifthen}                                           %%
\usepackage{lscape}     
\usepackage{multicol}
% \usepackage{enumerate}
\DeclareMathOperator*{\Res}{Res}
\renewcommand\thesection{\arabic{section}}
\renewcommand\thesubsection{\thesection.\arabic{subsection}}
\renewcommand\thesubsubsection{\thesubsection.\arabic{subsubsection}}
\renewcommand\thesectiondis{\arabic{section}}
\renewcommand\thesubsectiondis{\thesectiondis.\arabic{subsection}}
\renewcommand\thesubsubsectiondis{\thesubsectiondis.\arabic{subsubsection}}
\hyphenation{op-tical net-works semi-conduc-tor}
\def\inputGnumericTable{}                                 %%
\lstset{
frame=single, 
breaklines=true,
columns=fullflexible
}
\begin{document}

\newcommand{\solution}{\noindent \textbf{Solution: }}

\graphicspath{{figures/}}
\title{AI1110 - Assignment 1}

\author{Kushagra Gupta \\ \normalsize CS21BTECH11033  \\ \Large ICSE 2018 Grade 10}
\date{}
\maketitle
\begin{flushleft}
\textbf{Q10-c:}
The angle of elevation from a point P of the top of a tower QR, 50m high is $60\degree$ and that of the
tower PT from a point Q is $30\degree$. Find the height of tower PT, correct to the nearest metre.

\begin{center}
\includegraphics[width=150pt]{figure.png}
\end{center}
\begin{table}[ht!]
    \input{tables/table-1}
    \caption{}
	\label{table:table1}
\end{table}
\solution In $\Delta$PQR, using basic trigonometric equation in a right-angled triangle, we know that,
\begin{align}
&\tan(\theta)=\frac{\text{perpendicular}}{\text{base}}\\
\hspace{30pt}&\Rightarrow\tan(\alpha) = \frac{h}{d} \\
&\Rightarrow d = \frac{h}{\tan(\alpha)} \\
&\Rightarrow d = \frac{50}{\tan(60\degree)}\, m \\
&[\because \alpha = 60\degree \, \& \, h = 50m] \\
&\Rightarrow d = \frac{50}{\sqrt{3}}\, m
\end{align} 

Now in $\Delta PQT$, $\beta = 30\degree$.

\begin{align}
\therefore \tan&(\beta) = \frac{h_2}{d} \\
&\Rightarrow h_2 = d \times \tan(\beta)\\
&\Rightarrow h_2 = d \times \tan(30 \degree)\\
&\Rightarrow h_2 = \frac{50}{\sqrt{3}} \times \tan(30 \degree) \, m \\
&[using (6)]\\
&\Rightarrow h_2 = \frac{50}{3}\, m
\end{align}

$\therefore h_2(PT)   \approx \fbox{17} $ metres after rounding off.


This can be verified by plotting QR , $\alpha$ and $\beta$ and approximating 
the length of PT.

\noindent\textbf{Output:}
The Output of the program used to verify the answer is given below:

\begin{figure}[h]
\includegraphics[width=252pt]{output.png}
\caption{Plot of the figure and calculated length}
\end{figure}

\end{flushleft}
\end{document}
