%%%%%%%%%%%%%%%%%%%%%%%%%%%%%%%%%%%%%%%%%%%%%%%%%%%%%%%%%%%%%%%
%
% Welcome to Overleaf --- just edit your LaTeX on the left,
% and we'll compile it for you on the right. If you open the
% 'Share' menu, you can invite other users to edit at the same
% time. See www.overleaf.com/learn for more info. Enjoy!
%
%%%%%%%%%%%%%%%%%%%%%%%%%%%%%%%%%%%%%%%%%%%%%%%%%%%%%%%%%%%%%%%

% Inbuilt themes in beamer
\documentclass{beamer}

%packages:
% \usepackage{tfrupee}
% \usepackage{amsmath}
% \usepackage{amssymb}
% \usepackage{gensymb}
% \usepackage{txfonts}

% \def\inputGnumericTable{}

% \usepackage[latin1]{inputenc}                                 
% \usepackage{color}                                            
% \usepackage{array}                                            
% \usepackage{longtable}                                        
% \usepackage{calc}                                             
% \usepackage{multirow}                                         
% \usepackage{hhline}                                           
% \usepackage{ifthen}
% \usepackage{caption} 
% \captionsetup[table]{skip=3pt}  
% \providecommand{\pr}[1]{\ensuremath{\Pr\left(#1\right)}}
% \providecommand{\cbrak}[1]{\ensuremath{\left\{#1\right\}}}
% %\renewcommand{\thefigure}{\arabic{table}}
% \renewcommand{\thetable}{\arabic{table}}      

\setbeamertemplate{caption}[numbered]{}

\usepackage{enumitem}
\usepackage{tfrupee}
\usepackage{amsmath}
\usepackage{amssymb}
\usepackage{gensymb}
\usepackage{graphicx}
\usepackage{txfonts}

\def\inputGnumericTable{}

\usepackage[latin1]{inputenc}                                 
\usepackage{color}                                            
\usepackage{array}                                            
\usepackage{longtable}                                        
\usepackage{calc}                                             
\usepackage{multirow}                                         
\usepackage{hhline}                                           
\usepackage{ifthen}
\usepackage{caption} 
\captionsetup[table]{skip=3pt}  
\providecommand{\pr}[1]{\ensuremath{\Pr\left(#1\right)}}
\providecommand{\cbrak}[1]{\ensuremath{\left\{#1\right\}}}
\renewcommand{\thefigure}{\arabic{table}}
\renewcommand{\thetable}{\arabic{table}}   
\providecommand{\brak}[1]{\ensuremath{\left(#1\right)}}

% Theme choice:
\usetheme{CambridgeUS}

% Title page details: 
\title{Assignment 7} 
\author{Kushagra Gupta}
\date{\today}
% \logo{\large \LaTeX{}}


\begin{document}

% Title page frame
\begin{frame}
    \titlepage 
\end{frame}

% Remove logo from the next slides
\logo{}


% Outline frame
\begin{frame}{Outline}
    \tableofcontents
\end{frame}



\section{Problem Statement}
\begin{frame}{Problem Statement}
    \begin{block}{13.4 Q7 [NCERT 12] } A coin is biased so that the head is 3 times as likely to occur as tail. If the coin is tossed twice, find the probability distribution of number of tails.
    \end{block}
\end{frame}



\section{Definitions}
\begin{frame}{Random Variable Definition}
In this experiment, there are two consecutive Bernoulli trials. Therefore, it is appropriate to define a Binomial Random Variable $X$ as under:


\begin{table}[ht!]
    \centering
    \input{tables/table1}
    \caption{Random Variable $X$}
	\label{table:table1}
\end{table}

  
\end{frame}

\begin{frame}{Random Variable Definition}
In each Bernoulli trial, let Binomial Random Variable $Y$ be as under:

\begin{table}[ht!]
    \centering
    \input{tables/table2}
    \caption{Random Variable $Y$}
	\label{table:table2}
\end{table}
\end{frame}

\begin{frame}{Probability Mass Function}
We know that,
\begin{block}{}
       \begin{align}
            &\pr{Y=0} = 3\times \pr{Y=1}\\
            &\pr{Y=0} + \pr{Y=1} = 1\\
            \therefore &\pr{Y=1} = \frac{1}{4} \text{\,and} \pr{Y=0} = \frac{3}{4}
       \end{align}
\end{block}
\end{frame}
\begin{frame}{Probability Mass Function}
    

The probability of success is $p=\frac{1}{4}$.

Therefore, the probability that $X$ maps to $i$ is given by:
\begin{block}{}
       \begin{align}
                \label{eq1}
           \pr{X=i} = \binom{2}{i} (1-p)^{2-i} p^i ,~ 0 \le i \le 2 
       \end{align}
\end{block}

The values for $i$ can be substituted in the above formula, and the graph of the PMF can be obtained.
\end{frame}


\begin{frame}{Cumulative Distribution Function}
The cumulative probability $ \pr{X \leq i}$ can be defined as under:

\begin{block}{}
\begin{align}
          \label{eq2}
       \pr{X \leq i} = \sum_{k=0}^{i} \binom{2}{k} (1-p)^{2-k} p^k ,~ 0 \le i \le 2
\end{align}
\end{block}

The values of $i$ can be substituted in the above equation, and the obtained values can be used to plot the CDF graph.

\end{frame}


\section{Solution}
\begin{frame}{Solution}
We have to find the probability distribution of the number of tails in the trials. So, plugging $i=0,1,2$ in equation \eqref{eq1}
\begin{align}
    \pr{X=i} = 
    \begin{cases}
        &\frac{9}{16}, i = 0 \\
        &\frac{3}{8}, i = 1\\
        &\frac{1}{16}, i = 2
    \end{cases}
\end{align}
\end{frame}

\section{Graphs}
\begin{frame}{PMF Graph}
The PMF graph is:
    \begin{figure}[!ht]
		\centering
		\includegraphics[width=\textwidth,height=5.5cm,keepaspectratio]{figures/figure1.png}
		\caption{Probability Mass Function}
		\label{fig1}
	\end{figure}
\end{frame}

\begin{frame}{CDF Graph}
The CDF graph is:
    \begin{figure}[!ht]
		\centering
		\includegraphics[width=\textwidth,height=5.5cm,keepaspectratio]{figures/figure2.png}
		\caption{Cumulative Distribution Function}
		\label{fig2}
	
\end{figure}
\end{frame}


\end{document}

